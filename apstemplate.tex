%% ****** Start of file apstemplate.tex ****** %
%%
%%
%%   This file is part of the APS files in the REVTeX 4 distribution.
%%   Version 4.1r of REVTeX, August 2010
%%
%%
%%   Copyright (c) 2001, 2009, 2010 The American Physical Society.
%%
%%   See the REVTeX 4 README file for restrictions and more information.
%%
%
% This is a template for producing manuscripts for use with REVTEX 4.0
% Copy this file to another name and then work on that file.
% That way, you always have this original template file to use.
%
% Group addresses by affiliation; use superscriptaddress for long
% author lists, or if there are many overlapping affiliations.
% For Phys. Rev. appearance, change preprint to twocolumn.
% Choose pra, prb, prc, prd, pre, prl, prstab, prstper, or rmp for journal
%  Add 'draft' option to mark overfull boxes with black boxes
%  Add 'showpacs' option to make PACS codes appear
%  Add 'showkeys' option to make keywords appear
%%\documentclass[aps,prl,preprint,groupedaddress]{revtex4-1}
\documentclass[5p]{elsarticle}%[prstab,11pt,showpacs]{revtex4-1}
\usepackage{graphicx}
\usepackage{booktabs}
\usepackage{amssymb}
\usepackage{amsmath}
\usepackage{epsfig}
\usepackage{color}
\usepackage{cleveref}

%\usepackage{multline}
%\documentclass[aps,prl,preprint,superscriptaddress]{revtex4-1}
%\documentclass[aps,prl,reprint,groupedaddress]{revtex4-1}

% You should use BibTeX and apsrev.bst for references
% Choosing a journal automatically selects the correct APS
% BibTeX style file (bst file), so only uncomment the line
% below if necessary.
%\bibliographystyle{apsrev4-1}
\begin{document}
% Use the \preprint command to place your local institutional report
% number in the upper righthand corner of the title page in preprint mode.
% Multiple \preprint commands are allowed.
% Use the 'preprintnumbers' class option to override journal defaults
% to display numbers if necessary
%\preprint{}
%Title of paper
%\title{Electron Cloud Effects in Coasting Beam with Nonlinear Space Charge}
\title{Electron Cloud Memory Effects}
% repeat the \author .. \affiliation  etc. as needed
% \email, \thanks, \homepage, \altaffiliation all apply to the current
% author. Explanatory text should go in the []'s, actual e-mail
% address or url should go in the {}'s for \email and \homepage.
% Please use the appropriate macro foreach each type of information
% \affiliation command applies to all authors since the last
% \affiliation command. The \affiliation command should follow the
% other information
% \affiliation can be followed by \email, \homepage, \thanks as well.
\author{F. Petrov}
%\email[]{Your e-mail address}
%\homepage[]{Your web page}
%\thanks{}
%\altaffiliation{}
\address[tudtemf]{Institut f\"ur Theorie Elektromagnetischer Felder (TEMF), Technische Universit\"{a}t Darmstadt,
Schlo{\ss}gartenstr. 8 64289 Darmstadt}
%\affiliation{}
%Collaboration name if desired (requires use of superscriptaddress
%option in \documentclass). \noaffiliation is required (may also be
%used with the \author command).
%\collaboration can be followed by \email, \homepage, \thanks as well.
%\collaboration{}
%\noaffiliation
\date{\today}
\begin{abstract}
Electron cloud is a concern for many modern and future accelerator facilities. There are a number of undesired effects attributed to the presence of electron clouds. Among them are coherent instabilities, emittance growth, cryogenic heat load, synchronous phase shift and pressure rise. In long bunch trains one can observe the emittance growth getting faster along the train. This coupled bunch effect is mainly due to the growing electron cloud density along the bunch train. In this paper we address other mechanisms that can lead to the coupled-bunch electron cloud effects.
\end{abstract}

% insert suggested PACS numbers in braces on next line
%\pacs{}
% insert suggested keywords - APS authors don't need to do this
%\keywords{}

%\maketitle must follow title, authors, abstract, \pacs, and \keywords
\maketitle

\section{Introduction}

\section{Long Range Wakefields due to Secondary Emission}

\section{Simulations}
	The numerical model is the same as in Ref.[Petrov2014].
	In this section we focus on the dipole region mainly. 
	In reality primary electrons are bound to the magnetic field lines.
	Moreover, secondary electrons move along approximately the same field lines as the primary ones.
	Thus the electrons stay approximately in one and the same plane and the approximation of 2D grid and 2D electron motion is rather valid.
	In contrast in field-free regions physical electrons can move freely in all the directions.
	Hence, in the gap between two consecutive bunches

\subsection{Simulations for Round Geometry}

\subsection{Simulations for Rectangular Geometry}

\subsection{Simulations for Realistic LHC Geometry}

\subsection{Conclusion and Outlook}

\begin{thebibliography}{16}%


\end{thebibliography}%



\end{document}
%
% ****** End of file apstemplate.tex ******

