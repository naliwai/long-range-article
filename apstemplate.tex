%% ****** Start of file apstemplate.tex ****** %
%%
%%
%%   This file is part of the APS files in the REVTeX 4 distribution.
%%   Version 4.1r of REVTeX, August 2010
%%
%%
%%   Copyright (c) 2001, 2009, 2010 The American Physical Society.
%%
%%   See the REVTeX 4 README file for restrictions and more information.
%%
%
% This is a template for producing manuscripts for use with REVTEX 4.0
% Copy this file to another name and then work on that file.
% That way, you always have this original template file to use.
%
% Group addresses by affiliation; use superscriptaddress for long
% author lists, or if there are many overlapping affiliations.
% For Phys. Rev. appearance, change preprint to twocolumn.
% Choose pra, prb, prc, prd, pre, prl, prstab, prstper, or rmp for journal
%  Add 'draft' option to mark overfull boxes with black boxes
%  Add 'showpacs' option to make PACS codes appear
%  Add 'showkeys' option to make keywords appear
%%\documentclass[aps,prl,preprint,groupedaddress]{revtex4-1}
\documentclass[5p]{elsarticle}%[prstab,11pt,showpacs]{revtex4-1}
\usepackage{graphicx}
\usepackage{booktabs}
\usepackage{amssymb}
\usepackage{amsmath}
\usepackage{epsfig}
\usepackage{color}
\usepackage{cleveref}

%\usepackage{multline}
%\documentclass[aps,prl,preprint,superscriptaddress]{revtex4-1}
%\documentclass[aps,prl,reprint,groupedaddress]{revtex4-1}

% You should use BibTeX and apsrev.bst for references
% Choosing a journal automatically selects the correct APS
% BibTeX style file (bst file), so only uncomment the line
% below if necessary.
%\bibliographystyle{apsrev4-1}
\begin{document}
% Use the \preprint command to place your local institutional report
% number in the upper righthand corner of the title page in preprint mode.
% Multiple \preprint commands are allowed.
% Use the 'preprintnumbers' class option to override journal defaults
% to display numbers if necessary
%\preprint{}
%Title of paper
%\title{Electron Cloud Effects in Coasting Beam with Nonlinear Space Charge}
\title{Electron Cloud Memory Effects}
% repeat the \author .. \affiliation  etc. as needed
% \email, \thanks, \homepage, \altaffiliation all apply to the current
% author. Explanatory text should go in the []'s, actual e-mail
% address or url should go in the {}'s for \email and \homepage.
% Please use the appropriate macro foreach each type of information
% \affiliation command applies to all authors since the last
% \affiliation command. The \affiliation command should follow the
% other information
% \affiliation can be followed by \email, \homepage, \thanks as well.
\author{F. Petrov}
%\email[]{Your e-mail address}
%\homepage[]{Your web page}
%\thanks{}
%\altaffiliation{}
\address[tudtemf]{Institut f\"ur Theorie Elektromagnetischer Felder (TEMF), Technische Universit\"{a}t Darmstadt,
Schlo{\ss}gartenstr. 8 64289 Darmstadt}
%\affiliation{}
%Collaboration name if desired (requires use of superscriptaddress
%option in \documentclass). \noaffiliation is required (may also be
%used with the \author command).
%\collaboration can be followed by \email, \homepage, \thanks as well.
%\collaboration{}
%\noaffiliation
\date{\today}
\begin{abstract}
Typically the electron cloud effects are simulated in two stages: electron-cloud build-up simulations and electron-cloud induced instability simulation. In the latter stage the clouds are usually refreshed after each beam-cloud interaction. In this note we study memory effects of the cloud within the 2D electrostatic particle-in-cell code.
\end{abstract}

% insert suggested PACS numbers in braces on next line
%\pacs{}
% insert suggested keywords - APS authors don't need to do this
%\keywords{}

%\maketitle must follow title, authors, abstract, \pacs, and \keywords
\maketitle

\section{Simulations}

The idea of the study is very simple(Fig.~\ref{offcentered}).
We begin with a typical electron cloud build-up simulation and wait until the saturation.
Several bunches before the train end we introduce an offset to the bunch.
This bunch sees the strongest transverse electric field.
Moreover, the electron cloud distribution changes slightly.
Latter bunches interact with this distorted cloud.


\begin{figure}[htb]
\centering
\includegraphics*[width=80mm]{data/offcentered.eps}
\caption{(Color) The arrangement of the bunches in the simulation.}
\label{offcentered}
\end{figure}

Figs.~\ref{wake_drift},~\ref{wake_dipole_x}, and ~\ref{wake_dipole_y} show the transverse electric field averaged over the centered bunch profile.
One can clearly see that the first bunch induces no transverse field.
The next off-centered bunch sees the strongest transverse electric field.
Centered bunches that follow the distorted bunch continue inducing the transverse electric field.


\begin{figure}[htb]
\centering
\includegraphics*[width=80mm]{data/round_drift_offset.eps}
\caption{(Color) Transverse electric field seen by the bunches in the train passing through the round field free section. At 75 m starts the offceneted bunch. Its offset is listed in the figure. Upper graph shows the field seen on the pipe axis near the offcentered bunch. Lower graph shows the transverse field seen by one bunch before and two bunches after the distorted bunch.}
\label{wake_drift}
\end{figure}


\begin{figure}[htb]
\centering
\includegraphics*[width=80mm]{data/round_dipole_xoffset.eps}
\caption{(Color) Horizontal electric field seen by the bunches in the train passing through the round field free section. At 75 m starts the offceneted bunch. Its offset is listed in the figure. Upper graph shows the field seen on the pipe axis near the offcentered bunch. Lower graph shows the transverse field seen by one bunch before and two bunches after the distorted bunch.}
\label{wake_dipole_x}
\end{figure}


\begin{figure}[htb]
\centering
\includegraphics*[width=80mm]{data/round_dipole_yoffset.eps}
\caption{(Color) Vertical electric field seen by the bunches in the train passing through the round field free section. At 75 m starts the offceneted bunch. Its offset is listed in the figure. Upper graph shows the field seen on the pipe axis near the offcentered bunch. Lower graph shows the transverse field seen by one bunch before and two bunches after the distorted bunch.}
\label{wake_dipole_y}
\end{figure}

However, the actual field seen by the offcentered bunch is significantly different from the field on the pipe axis (Figs.~\ref{vs_dip_y},~\ref{vs_dip_x} and~\ref{vs_dri_y}).
\begin{figure}[htb]
\centering
\includegraphics*[width=80mm]{data/vs_dip_y.eps}
\caption{(Color) Comparison of the field on the pipe axis (upper) and on the bunch axis (lower) in dipole with vertically offcentered bunch.}
\label{vs_dip_y}
\end{figure}

\begin{figure}[htb]
\centering
\includegraphics*[width=80mm]{data/vs_dip_x.eps}
\caption{(Color) Comparison of the field on the pipe axis (upper) and on the bunch axis (lower) in dipole with horizontally offcentered bunch.}
\label{vs_dip_x}
\end{figure}

\begin{figure}[htb]
\centering
\includegraphics*[width=80mm]{data/vs_dri_y.eps}
\caption{(Color) Comparison of the field on the pipe axis (upper) and on the bunch axis (lower) in drift with vertically offcentered bunch.}
\label{vs_dri_y}
\end{figure}

\begin{center}
\begin{table}
    \caption{Simulation parameters for LHC-type bunches.}
    \begin{tabular}{ | l | c |}
    \hline
    Bunch length, $\sigma_{z}$ / m & 0.1 \\ \hline
    Bunch radius, $\sigma_{r}$ / m & $10^{-3}$ \\ \hline
    Bunch intensity & $1 \cdot 10^{11}$  \\ \hline
    Bunch spacing / ns & 25  \\ \hline
    Pipe radius, $R_{p}$ / m & $2\cdot10^{-2}$ \\ \hline
    Magnetic field, B / T & 0.1  \\ \hline
    Maximum SEY, $\delta_{max}$ & 1.4 \\ \hline
    Energy of $\delta_{max}$, $W_{sey,max}$ / eV & 250  \\ \hline
    Rediffusion probability & 0.7  \\
    \hline
    \end{tabular}
\label{table_1}
\end{table}
\end{center}

These simulations indicate clearly the memory effect of the electron cloud at 25 ns bunch spacing.

\section{Electron cloud memory as a function of SEY}

In this section we analyze how the features of the electron cloud field depend on the secondary emission curve.
We have chosen two ways to perform this study. 
One way is to have a fixed SEY curve an preform a scan over bunch intensities. 
Second way is to have a fixed intensity but varying position of the SEY maximum.
The first approach is closer to reality because one typically have a certain pipe material with almost fixed SEY parameters. To change the intensity is not a problem in most of the accelerators. 
However, at different intensities the dynamics of the pinch changes significantly and the comparison of the fields for different intensities becomes complicated.
The second approach in this sense is better, because the dynamics of the pinch is basically the same.
The effect of the pipe geometry on the transverse field is studied in this section as well.

\section{Appendix: Line density}
This section shows the evolution of the line density after the distorted bunch.
\begin{figure}[htb]
\centering
\includegraphics*[width=80mm]{data/dipole_line_y.eps}
\caption{(Color) Line density at saturation for the vertically offcentered bunch in the dipole section.}
\label{line_dip_y}
\end{figure}

\begin{figure}[htb]
\centering
\includegraphics*[width=80mm]{data/dipole_line_x.eps}
\caption{(Color) Line density at saturation for the horizontally offcentered bunch in the dipole section.}
\label{line_dip_x}
\end{figure}

\begin{figure}[htb]
\centering
\includegraphics*[width=80mm]{data/drift_line_y.eps}
\caption{(Color) Line density at saturation for the vertically offcentered bunch in the dridt section.}
\label{line_dri_y}
\end{figure}
\begin{thebibliography}{16}%


\end{thebibliography}%



\end{document}
%
% ****** End of file apstemplate.tex ******

